\documentclass[12pt]{article}  
\usepackage{graphicx}
\usepackage{geometry}   %设置页边距的宏包
\usepackage{algpseudocode}
\usepackage{comment}
\usepackage{amsmath, amssymb, amsthm}
\usepackage{enumerate}
\usepackage{enumitem}
\usepackage{framed}
\usepackage{verbatim}
\usepackage{microtype}
\usepackage{kpfonts}
\usepackage{multicol}
\usepackage{amsfonts}
\usepackage{array}
\usepackage{color}
\usepackage{pgf,tikz}
\usepackage{mathtools}
\usetikzlibrary{automata, positioning, arrows}
\usepackage{wrapfig}
\newcommand{\solu}{{\color{blue} Solution:}}
\newcommand{\overbar}[1]{\mkern 1.5mu\overline{\mkern-1.5mu#1\mkern-1.5mu}\mkern 1.5mu}
\newcommand{\Ib}{\mathbf{I}}
\newcommand{\Pb}{\mathbf{P}}
\newcommand{\Qb}{\mathbf{Q}}
\newcommand{\Rb}{\mathbf{R}}
\newcommand{\Nb}{\mathbf{N}}
\newcommand{\Fb}{\mathbf{F}}
\newcommand{\Z}{\mathbf{Z}}
\newcommand{\Lap}{\mathcal{L}}
\newcommand{\Zplus}{\mathbf{Z}^+}
\newcommand{\indep}{\perp \!\!\! \perp}
\DeclareMathOperator*{\argmin}{\arg\min}
\DeclareMathOperator*{\argmax}{\arg\max}
\geometry{left=1.5cm,right=1.5cm,top=1.5cm,bottom=1.5cm}  %设置 上、左、下、右 页边距

\title{DS4400 HW2}
\author{Xin Guan}
\date{}


\begin{document}
    \begin{enumerate}
        \item \begin{enumerate}
            \item \solu
            
            we write $P(X = x) = \theta^x(1-\theta)^{(1-x)}$\\
            $P(D | \theta) = P(X_1 = x_1, X_2 = x_2 \dots, X_N= x_N)\\
            = \prod\limits_{x = 1}^{N}\theta^x_i(1-\theta)^{(1-x_i)}\\
            = \theta^{N_1}(1-\theta)^{N_0}$\\
            Let $J(\theta) = logP(D | \theta) = N_1log(\theta) + N_0log(1 - \theta)$\\
            $\frac{\partial J(\theta)}{\partial \theta} = \frac{N_1}{\theta} - \frac{N_0}{1 - \theta}$.
            Let $\frac{\partial J(\theta)}{\partial \theta} = 0$.\\
            Then $\hat{\theta} = \frac{N_1}{N_0 + N_1} = \frac{N_1}{N}$\\
            Therefore, the maximum likelihood solution is $\hat{\theta} = \frac{N_1}{N}$

            \item \solu 
            
            $p(D | \theta) \times p(\theta) =  \left\{
                \begin{array}{rcl}
                0.2\cdot 0.6^{N_1} \cdot 0.4^{N_0}  & & \theta = 0.6 \\
                0.8\cdot 0.8^{N_1} \cdot 0.2^{N_0} & & \theta = 0.8 \\
                0 & & \text{otherwise}
            \end{array}
            \right.$\\
            $\frac{P(D|0.6)P(0.6)}{P(D|0.8)P(0.8)} = \frac{1}{4}(\frac{3}{4})^{N_1}(2)^{N_0} = 3^{N_1}4^{-N_1-1}2^{N_0} = 2^{N_1log_2^3}2^{-2N_1 - 2}2^{N_0} = 2^{N_0 - (2-log_2^3)N_1 - 2}$\\
            Therefore, when $\frac{P(D|0.6)P(0.6)}{P(D|0.8)P(0.8)} \ge 1:$\\
            $N_0 -(2-log_2^3)N_1 -2 \ge 0 \Rightarrow N_0 \ge (2-log_2^3)N_1 +2$\\
            Therefore, $\hat{\theta} = \left\{
                \begin{array}{rcl}
                0.6 & & N_0 \ge (2-log_2^3)N_1 +2 \\
                0.8 & & N_0 < (2-log_2^3)N_1 +2
            \end{array}
            \right.$
        \end{enumerate}

        \item \begin{enumerate}
            \item \solu 
        
            $\theta^y_j = \left\{
                \begin{array}{rcl}
                    \frac{3}{7} & & j = 0\\
                    \frac{4}{7} & & j = 1
                \end{array}
            \right.$\\
            $\theta^{x_\ell | y}_{\overbar{x}_\ell | j} = \left\{
                \begin{array}{rcl}
                    \frac{1}{3} & & x_1 = 1, j = 0\\
                    \frac{2}{3} & & x_1 = 0, j = 0\\
                    \frac{1}{3} & & x_2 = 1, j = 0\\
                    \frac{2}{3} & & x_2 = 0, j = 0\\
                    \frac{1}{2} & & x_1 = 1, j = 1\\
                    \frac{1}{2} & & x_1 = 0, j = 1\\
                    \frac{1}{2} & & x_2 = 1, j = 1\\
                    \frac{1}{2} & & x_2 = 0, j = 1
                \end{array}
            \right.$

            \item \solu 
            
            $P(y = 0 | x_1 = 0, x_2 = 1)\\
            = \frac{P(x_1 = 0, x_2 = 1 | y = 0)P(y = 0)}{P(x_1 = 0, x_2 = 1)}\\
            = \frac{P(x_1 = 0| y = 0)P(x_2 = 1| y = 0)P(y = 0)}{P(x_1 = 0, x_2 = 1)}\\
            = \frac{\theta^{x_1 | y}_{0 | 0}\theta^{x_2 | y}_{1 | 0}\theta^y_0}{P(x_1 = 0, x_2 = 1)}\\
            = \frac{2}{3} \cdot \frac{1}{3} \cdot \frac{3}{7}  / \frac{2}{7}\\
            = \frac{1}{3}$
        \end{enumerate}

        
    \end{enumerate}    
\end{document}
